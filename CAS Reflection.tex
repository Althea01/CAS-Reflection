\documentclass[11pt,a4paper,titlepage]{article}
\usepackage[utf8]{inputenc}
\usepackage{amsmath}
\usepackage{amsfonts}
\usepackage{amssymb}
\usepackage{graphicx}
\usepackage{booktabs}
\usepackage{siunitx}
\author{Lily}
\date{22/6/15}
\title{CAS Reflection}

\begin{document}
\maketitle

\tableofcontents
\clearpage

\section{Pragmatists}
\subsection{Pragmatists 1}
I would like to divide up what I have done this semester into three sections: How we have managed to sell our boxes and what we have learned from the experience; what we have planned for the future development of our club and other ideas we have thought of; The whole process of holding the design competition and the results and the possible future for this competition.

For the first part, we have The Boxes left from the previous semester and we decided to sell them as soon as possible so that we could concentrate on our next program. However, things appear to be more difficult than what we have thought. Our strategy is to go to each class and share the videos we have made last semester and show students how our boxes can be used. Truth is when we were in classes of grade 10 students, they did not appear to be as interested and they already had a locker right in their classrooms so our boxes seemed to be in less use. We were at first very disappointed by the result but then we didn’t give in to this challenge. We hoped that we can help the students and so we started to start from students of lower grades and we went to grade 8. Surprisingly, our boxes seemed more attractive there and we managed to sell all our boxes within two weeks’ time.

During the process of selling the boxes, a very interesting that happened is that we used WeChat as a way of propaganda and in order to sell our boxes better, we think of a way to help students better and let more students know about our club so we told the students who have received our boxes that we would help them repair their boxes when they are broken if they have added our WeChat account as friend and have paid attention to our WeChat. The method appeared to be very effective, students were more interested in our club and what we have done until then.

Another thing that I have remembered is when one day, we wanted to inform all students and teachers in our school of our boxes, I used the broadcast and for such a shy person like me, I think that it is such a challenge for me to think of this idea and also do it myself. I would feel nervous in front of only a few people that I’m not familiar with, yet I was speaking to the whole school and I thought that at that time, I shall experience the bravest moment of my life.

We have also utilized the opportunity of club demonstration and introduced our boxes to students who have passed by our club and whenever there is a chance, we would be the first club that appear.

When I reflect this experience of selling boxes, I feel proud of what we have achieved though the amount of money we earned might not have been much and have been used up by us within a month. I thought I have learned a lot and grow a lot in the process and I am thankful to the club and every club member.

\subsection{Pragmatists 2}
I would like to divide up what I have done this semester into three sections: How we have managed to sell our boxes and what we have learned from the experience; what we have planned for the future development of our club and other ideas we have thought of; The whole process of holding the design competition and the results and the possible future for this competition.

For the second part, we have planned to have more tenth graders in our club so that we will have some students who are interested in creative design that can inherit our club. Also, we have decided to continue making designs come true and sell them so that they can be more useful and make life more convenient. We will continue to communicate some ideas that we have thought of in our daily life and discuss some possible designs.

What we have thought of that is our new project that is to improve the situation when students cannot find a place to put down their books and other things when hurrying to the restroom. We have checked the restrooms to make sure of the structure inside so that we can either find a table or design a board ourselves to put in the restrooms so that it will be more convenient for the students. We are now deciding to put a smaller shelf-like thing in two of the restrooms to see if it will be more beneficial or rather becomes an obstacle.

The result turned out not as what we have expected. Students do not think of the tables as inconvenient but rather useful but our school does not allow us to put such tables in the restrooms so as a result, our first design was sadly abandoned. Thought this experience we learned that before we took any actions, we should first learn to communicate well to make sure that every step we take is useful and not just something that would fail in the very end anyway.

This semester, we have also went to the first creator group in China that is a very professional organization for designing all kinds of products. Since there is an interesting lecture that is mostly related to computer science, four of our club members went to the place. Initially, our purpose is to see how professional creator organizations operate so that we can learn from their experiences, also, when we arrived, we thought of finding sponsors for the competition that we plan to hold. The lecture tells about panorama and other techniques that is related to the visual world and we got the chance to try on the special made glasses that can let us truly feel that we are on a roller coaster.

We have learned much from the experience and believed that one day we can be better than they are. We will be more influential than they are and will be able to get more people to be interested in creative design and help to make people’s life more convenient through proper collaboration.

\subsection{Pragmatists 3}
I would like to divide up what I have done this semester into three sections: How we have managed to sell our boxes and what we have learned from the experience; what we have planned for the future development of our club and other ideas we have thought of; The whole process of holding the design competition and the results and the possible future for this competition.

For the third part, we have already held the competition and all the processes, including the planning section, the competition section and what we have done after the competition are very important for reflection.

In the very beginning we mainly discussed about how to hold the contest for creative design. We decided to first hold a contest in school with about 50 students that form several groups and we intended to let the several groups that each consists of two to three students compete with each other by solving problems in daily life that requires inventing new creative inventions. They needed to hand in their design with both graphs and words to explain their ideas and the winner will be able to get the award from our club.

We discussed about the five questions we have thought of and the research have done about the five of them and figured it is too easy to think of two of them so we cancelled those. And then we thought of the possibility of adding a problem that requires participants to create an app that is useful in life. We also decided to put off our competition for one week as in the mid-term week, students will probably not notice our posters and so it will be hard for us to do propaganda.

We then held the Pragmatist contest that asks students to make inventions for solving problems in everyday life. More than 10 teams participated in the contest and most of them chose the the problem that asked them to think of way to keep our books open when we are trying to write an essay on the computer and can not spare our hands to hold the books. We planned to discuss their works on Friday and decide their prize in the contest on that day and probably give out the prizes to them within the next week.

We review the designs and grade them together with our club advisor. We develop different criteria that emphasize four areas: creativity, function, practicality and communication. For creativity we each give a score ranging from zero to twenty and take average. For function we evaluate how effectively the product solves the problem we provide and if it has extensive functions. For practicality we evaluate production difficulty and the product’s durability. And then, we also give score based on how effectively the designers communicate their message through text and diagrams.

After all prizes were given I used 3D modeling to make a model for the first prize design in order to see if this is effective and serve as a commemoration. I have spent a lot of time on this because 3D modeling was almost a new skill to me as I haven’t practiced in months and so I was glad that I finally succeeded.

\section{Computerization}
This semester, our club focuses on recommending our website to more students and teachers with the hope that this can benefit them when posting homework. We also started our plan to make the iOS version and Android version so that students and teachers can receive notifications and can know the homework at once without the need to check the website constantly. I have already held lectures to my fellow classmates to inform them of how to use our SAM system and have also sent e-mail instructions to teachers and have included both how to post homework and how students can enroll in different classes with the class id and check the homework.

I have gained the awareness of my ability to communicate with others and give out speeches. Because we need to be really convincing in order to get the first few users to try our website and use them as the only way of posting homework consistently. When I try to persuade teachers and students of the advantages of using our website, I realized the importance of being able to communicate my thoughts to others and structure my speech in a organized way so that the ones who listened to my speech can understand better.

I have undertaken new challenges because before we have finished our website, I thought the most difficult part should be programming the website but now I find it even harder to get people to use our website. I have went to see the grade leader and discussed with her the possibility of informing all students and teachers of our website on a meeting so that it can be more effective. However, she refused me and said that our system cannot be official website so it is not proper to do propaganda on a formal meeting. We are still trying to find ways to convince the grade leader of grade 9 now and have been waiting for Ariel to inform us of any information related.

I have realized my initiation through the process of thinking of a unique way to introduce our website to others. I used to be a very shy person but because of this experience, it requires me to go find teachers every other day and even teachers that I do not know before once a week to get comments on our websites. I need to appear more confident in order to strike them. Thus I have think of various ways of persuading them according to the courses that they teach and their own conditions. 

Collaboration is also very important as it is so hard for either of us alone to get hold of every teacher in our grade and each of us specialize in different areas so it is quite essential for us to divide up the works and better fulfill our goals. I am in charge of about ten teachers and my mission is to get all of them to use our system and until now, I have succeeded in letting all of them feel that  our system is necessary and beneficial for both students and teachers.

\section{Conversation}
I have sent an e-mail to the teacher to summarize what I have accomplished in this short semester. 

Pragmatists: 
This is my personal project and in this semester, I have managed to sell all the The Boxes we have left the last semester and the number of boxes was even in shortage. Then, we used the money we earned and the money I have gotten from our sponsor to hold a competition that asked students to think of creative ideas to solve some problems in daily life to make life more convenient. The competition was a great success to me and around ten groups of students and even a teacher have participated and we have decided the first, second and third prize. Then, I used 3D modeling to make the design of first prize though the result did not turned out as what we have expected.

Computerization:
For computerization, our club focuses on recommending our website to more students and teachers with the hope that this can benefit them when posting homework. We also started our plan to make the iOS version and Android version so that students and teachers can receive notifications and can know the homework at once without the need to check the website constantly. I have already held lectures to my fellow classmates to inform them of how to use our SAM system and have also sent e-mail instructions to teachers and have included both how to post homework and how students can enroll in different classes with the class id and check the homework.

CrunchyZoo:
I have mainly attended the national competition of linguistics and won the second prize. And the whole semester was about me learning skills related to how to solve the problems and how some languages are originated. 

ChemSquare:
I haven’t really attend the club this semester and haven’t been able to find a way to quit the club.

Ice Hockey:
I have practiced a lot for Ice Hockey up till now and the most exciting thing that happened this semester shall be that I have been a judge for an international competition and that I was very honored to talk to the national athletes face to face.

\section{CrunchyZoo}
This semester for CrunchyZoo, I have mainly participated in the IOLing program and I have learned much from both the preparation process and the competition itself when I went to Beijing to attend the final round. After a total of twelve hours of competing with other students overall, I was awarded the second prize and that served as an encouragement to me.

I have definitely gained awareness of my ability to solve problems logically as the competition is designed for this. And when I was in Beijing, I heard a lecture about how linguistics get to know a language. They joined the tribe for one and a half year and they gather information about their vocabulary, grammar and many other things, then they find the similarities and differences between sentence and sentence, word and word so they finally get to guess the meaning of each word and learn the sentence structure. They need to stay for at least one and a half year because persistence is what lead to success and most festivals are held every year. I can also see how logic is important and the process of us solving problems is just the same as the process they learn a new language and note it down to preserve it.

I have also seen global value in the experience since the competition has included languages from all around the world and many of them are close to extinction thus to preserve them reflects a recognition of global value. IOL serves more like a chance for students to get a big picture of how linguistics work in advance and students might gain interest and would also join the work of gathering information about languages all around the world so that one day, even if no one on earth is speaking a certain language, we still have books about them that we can learn the language again and so the language will not disappear.

I think to participate and solve the problems is challenge for me. I used to think that the IOL problems are very easy to solve since I was confident about my logical thinking. However, now I realized that IOL does not only require logical thinking but also the ability to learn quickly and be imaginative and have some necessary information. If I have never seen languages that do not use the decimal system, it will cost me more time to think of this possibility and thus to solve a problem. It is true that imagination is based on what we already know and the more we know, the more imaginative we shall be.

Most importantly, as I have mentioned before, commitment is a very important part of the whole IOL process. To be successful in solving the problems, one need to have patience so that we won’t stop trying to easily when a method seems to be wrong, but will keep going so that we might prove it right. Sometimes, we need to try tens or hundreds of possibilities and we know that one of them must be correct, however some students just don’t have the perseverance. I believe that in the process of practicing, I have finally gained the perseverance that I need for life.

\section{ChemSquare}
This semester, we didn’t have as much experiments as we did in the previous semester
because at the beginning of this semester, the few of us had a talk about what we should do for our club. It is partly because I am familiar with the club leader, but we discussed a lot about how to improve our club and identified several problems of our club, including the fact that in the process of doing chemical experiments, we weren’t actually learning anything if we just have the procedure and the reaction formula without knowing the mechanics of each experiment. So this is the reason why we are holding so many lectures this semester and we asked club members to give out presentations. This is a really helpful process for us to learn from each other and know more about knowledge we haven’t learned in class.

We have done several chemistry experiments this semester which includes producing silver, making soaps and making bubbles which in Chinese we called “Elephant toothpaste” because the bubbles appear suddenly when the reaction happens and look so much like the trunk of an elephant but have the appearance of toothpaste. I have felt that this is a great chance for me to do experiments that I cannot do on my own and since some parts are missing in the handout the club leader gave us, we can have the opportunity to try out different ratio of reactants to see what will happen and explore by ourselves to further help us develop innovation. Also, some experiments are a bit too complicated for a single person to complete thus we can find a partner, learn to collaborate and feel the joy of getting a successful result when the experiment is over and share the joy with another person or people in the group.

What also strikes me when reflecting on the experience I have in ChemSquare is that when only three of us are attending one meet-up of the club, we cannot discuss anything broad because that way the other members won’t get a chance to know what we have discussed, so we thought about the future of the club and as a matter of fact, we have
once attempted to include more service in our club. Our club is mostly based on Creativity but we think that just letting the club members learn things about Chemistry isn’t enough, and we really hoped to find another way to let normal people, even the ones that haven’t learned Chemistry before, benefit from the process of knowing more about Chemistry. So then we thought of making a booklet to let people know that some “common knowledge” of Chemistry in real life is just myth and we wanted to do experiments and include our results in the booklet to show people that some ideas about Chemistry are totally wrong. We thought of the possibility of giving the booklet out to primary school students and middle school students.

% Table generated by Excel2LaTeX from sheet 'π§◊˜±Ì1'
\begin{table}[S]
  \centering
  \caption{Add caption}
    \begin{tabular}{cc}
    \toprule
    2     & 21.212 \\
    3     & 234.1 \\
    \midrule
    askhdgk & hjkashd \\
    asdh  & asd \\
    \bottomrule
    \end{tabular}%
  \label{tab:addlabel}%
\end{table}
\end{document}